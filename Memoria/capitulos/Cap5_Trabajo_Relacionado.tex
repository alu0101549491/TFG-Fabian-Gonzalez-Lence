% ---------------------------------------------------
% Trabajo Final de Grado
% Author: Fabián González Lence <alu0101549491@ull.edu.es>
% Chapter: Related Work
% ----------------------------------------------------

\cleardoublepage
\chapter{Estado del Arte} \label{chap:Related_Work}

\begin{comment}
    Reseñar/Revisar aquí algunos de los trabajos más relevantes (related work) que hacen cosas similares a las que nosotros queremos hacer.\\
    
    Algunos de estos trabajos parecen relevantes.
    En cualquier caso, tener en cuenta la fecha de publicación de cada uno de ellos.
    Los más recientes sería conveniente revisarlos y a partir de ellos hallar otros trabajos relacionados.
    
    \begin{itemize}
        \item Maleki \& Zhao (2024) \cite{Maleki:2024:PCG} hacen una revisión reciente sobre generación procedural de contenido en videojuegos.
         Dado lo reciente de este trabajo, yo empezaría por estudiar este.
    
        \item Este trabajo \cite{Latif:2022:CEP} (de 2022) presenta una revisión crítica de herramientas y algoritmos para generación procedural de terrenos.
    
        \item Este \cite{Hendrikx:2013:PCG} (año 2013) es un Survey general sobre generación procedural de contenido en juegos.
    
        \item Este otro \cite{Raffe:2012:SPT} (2012) es otro survey, en este caso específico sobre generación procedural de terreno con algoritmos evolutivos.
    
        \item Otra revisión más \cite{Smelik:2014:SPM} (2014) sobre generación procedural de mundos virtuales.
    \end{itemize}
\end{comment}

La integración de la inteligencia artificial en la ingeniería del software ha evolucionado rápidamente con la aparición de los LLMs. Diversas investigaciones recientes coinciden en que estos modelos están transformando la forma en que se concibe, desarrolla y mantiene el software, al permitir automatizar tareas que tradicionalmente requerían intervención humana directa.\\

Wang y Chen \cite{Wang:2023:LLMReview} revisan el panorama actual de la generación de código con LLMs, destacando que estos modelos han demostrado una notable capacidad para comprender y producir código funcional en múltiples lenguajes de programación. Sin embargo, subrayan una carencia importante: la evaluación sistemática de la calidad del código no ha mostrado un avance al mismo ritmo que las aplicaciones prácticas, dejando abierta la necesidad de métodos de análisis más completos que consideren múltiples características de calidad más allá de la corrección funcional.\\

En la misma línea, Turunen y Fagerholm \cite{Turunen:2023:FromIdeas} exploran la automatización de la creación de aplicaciones a partir de ideas o especificaciones textuales, mostrando cómo herramientas basadas en agentes de IA (como AutoGPT o Microsoft AutoGen) permiten distribuir las tareas del ciclo de vida del software entre distintos roles inteligentes que colaboran mediante conversaciones. Estos autores apuntan que el futuro de la ingeniería del software podría pasar por entornos de agentes de IA colaborativos, en los que cada modelo actúe como desarrollador, tester, o gestor de requisitos, transformando radicalmente los flujos de trabajo tradicionales.\\

Por su parte, Davide Tosi \cite{Tosi:2024:CodeQuality} realiza una evaluación empírica de la calidad del código generado por distintos LLMs (GPT-3.5, GPT-4 y Google Bard), concluyendo que, aunque estos modelos logran resultados funcionales y coherentes en la mayoría de los casos de prueba, aún requieren supervisión constante de expertos en software y una correcta definición de requisitos para alcanzar una calidad excepcional. Además, observa que el código generado tiende a presentar una complejidad moderada-alta, junto con algunos code smells, lo que subraya la importancia de la revisión humana, incluso cuando el código funciona correctamente.\\

De modo complementario, Khade y Sambhe \cite{Khade:2025:AIinSD} amplían la perspectiva al examinar la aplicación de la IA no solo en la generación de código, sino también en pruebas automatizadas, mantenimiento y seguridad. Enfatizan que, más que reemplazar a los ingenieros de software, la IA está aumentando sus capacidades al automatizar tareas tediosas, permitiéndoles enfocarse en el diseño de alto nivel y la resolución de problemas complejos, aunque esto requiere una cooperación equilibrada entre humanos y máquinas.\\

En conjunto, estos estudios reflejan un campo de rápida expansión, pero todavía fragmentado. Las investigaciones actuales se centran mayoritariamente en el rendimiento individual de los modelos en tareas específicas, mientras que el potencial de colaboración entre múltiples IAs especializadas a lo largo del ciclo del desarrollo de software sigue poco explorado. Precisamente en este vacío se enmarca el presente Trabajo de Fin de Grado, que propone analizar cómo distintos agentes de IA pueden cooperar de forma complementaria para desarrollar software de manera integrada, evaluando la coherencia, calidad y eficacia del producto resultante.




