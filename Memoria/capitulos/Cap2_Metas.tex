% ---------------------------------------------------
% Trabajo Final de Grado
% Author: Fabián González Lence <alu0101549491@ull.edu.es>
% Chapter: Goals 
% ----------------------------------------------------


\chapter{Objetivos} \label{chap:Goals}

El objetivo principal de este TFG es desarrollar cinco aplicaciones 
utilizando \ac{IA} generativa:

\begin{enumerate}
  \item \ac{HANGMAN}: Un juego del ahorcado implementado como \ac{SPA}.
  \item \ac{PLAYER}: Un reproductor de música web interactivo.
  \item \ac{BALATRO}: Un juego de cartas inspirado en Balatro.
  \item \ac{CARTO}: Sistema de gestión de proyectos cartográficos.
  \item \ac{TENNIS}: Aplicación para la gestión de torneos de tenis.
\end{enumerate}

Posteriormente, analizaremos cómo \ac{HANGMAN} implementa el patrón \ac{MVC}.\\





Este capítulo se escribe normalmente al final de la realización del trabajo. 
En ese momento se tendrá una perspectiva más clara de lo que se ha pretendido hacer y de lo que realmente se ha hecho.

Los objetivos específicos a conseguir en el desarrollo de este software son los siguientes:

1.Creación del anteproyecto. Elaboración de un documento donde se expone una
introducción, antecedentes y estado actual, así como un listado de tareas a realizar
y el plan de trabajo establecido para la realización de este Trabajo Fin de Grado.

2.Creación y configuración de la estructura principal del proyecto. Creación de los di-
rectorios en los que se alojará todo el código, distribuido en los directorios frontend,
back-end y algorithm.

3.Creación de la API-REST. Desarrollo de un back-end capaz de comunicar la interfaz
de usuario con una base de datos como lo es MongoDB, haciendo uso del lenguaje
de programación Java y su framework Spring.

4.Creación de la interfaz de usuario. Desarrollo de un front-end desde el que el usuario
pueda interactuar con la aplicación. Para el desarrollo de esta interfaz se hizo uso
del framework Vue.js y Vuetify.

5.Implementación e integración en la aplicación del BAP. Implementación de un
módulo, aparte de los dos anteriores, capaz de resolver el roblema de gestión de
atraques (BAP).

6.Creación de una pantalla de resultados. Desarrollo de una pantalla en la que se
muestre de manera gráfica y detallada la salida obtenida al aplicar el BAP.

7.Redacción de la memoria. Se describe el Trabajo Fin de Grado desarrollado.


This document summarizes the research and development work carried out by the student in the achievement of his Final Degree Project (\textit{Trabajo de Fin de Grado}, TFG), which will conclude his studies for the degree \textit{Grado en Ingeniería Informática} at the \textit{Escuela Superior de Ingeniería y Tecnología} at the University of La Laguna (ULL).

This project has the following main goals:

\begin{enumerate}
    \item A first objective has been ... 
    \item Another key objective is ... 
    \item The implementation and ...
    \item Additionally, the student will apply....
    \item Finally, after thorough experimentation and evaluation, the student is expected to deliver...
\end{enumerate}
