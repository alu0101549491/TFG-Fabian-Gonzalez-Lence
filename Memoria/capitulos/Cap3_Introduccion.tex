% ---------------------------------------------------
% Trabajo Final de Grado
% Author: Fabián González Lence <alu0101549491@ull.edu.es>
% Chapter: Introduction 
% ----------------------------------------------------

\cleardoublepage
\chapter{Introduction} \label{chap:Introduction}
Introducción al lector en el problema que se propone resolver.\\

En la última década, los avances en el campo del aprendizaje profundo (\textit{deep learning}) y del procesamiento del lenguaje natural han propiciado la aparición de los denominados \textbf{modelos de lenguaje de gran tamaño} (\textit{Large Language Models, LLMs}), capaces de comprender y generar texto con un grado de coherencia y precisión cada vez mayor, abriendo nuevas posiblidades en diversos ámbitos, entre ellos la \textbf{generación automática de código}, una de las áreas de mayor interés y controversia en la ingeniería informática contemporánea.\\

\noindent El presente \textbf{Trabajo de Fin de Grado} tiene como objetivo principal \textbf{evaluar las capacidades de múltiples \textit{LLMs} en la generación de código} dentro del ámbito de las aplicaciones web, analizando la calidad, eficiencia y corrección de los resultados obtenidos frente a los estándares de desarrollo software actuales. Para ello, se estudiarán modelos representativos de la actualidad como \textbf{ChatGPT}, \textbf{Claude}, \textbf{Perplexity} y \textbf{GitHub Copilot}, con el propósito de determinar su rendimiento y fiabilidad en contextos de programación real, en donde se encargarán de construir cada proyecto desde cero casi en su totalidad.\\

\noindent La metodología adoptada se basa en la elaboración de un conjunto de \textbf{proyectos prácticos} de distinta complejidad, desde los que se evaluará el desempeño de cada uno de los modelos para generar código funcional, mantenible y estructurado conforme a los principios de la programación orientada a objetos, el diseño modular y las buenas prácticas de desarrollo web. Esta evaluación permitirá establecer un punto de comparativa entre cada uno de los modelos, destacando \textbf{fortalezas}, \textbf{debilidades} y \textbf{posibles áreas de mejora}.\\

\noindent Este trabajo busca contribuir al entendimiento del papel que desempeñan actualmente los modelos de lenguaje en la automatización de tareas de programación, así como reflexionar sobre su impacto en la \textbf{formación de futuros ingenieros informáticos} y en los \textbf{cambios que pueden ocasionar en el proceso de desarrollo de software}. En definitiva, con este estudio se busca ofrecer una visión crítica y fundamentada sobre el grado de madurez de estas herramientas, su aplicabilidad práctica y las implicaciones éticas y profesionales derivadas de su uso en entornos reales de programación de forma generalizada.\\

Colocar al final del capítulo un apartado
\textbf{Estructura del documento} en el que se explique el contenido de cada capítulo

